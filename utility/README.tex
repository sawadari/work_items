% This LaTeX was auto-generated from MATLAB code.
% To make changes, update the MATLAB code and export to LaTeX again.

\documentclass{article}

\usepackage[utf8]{inputenc}
\usepackage[T1]{fontenc}
\usepackage{lmodern}
\usepackage{graphicx}
\usepackage{color}
\usepackage{hyperref}
\usepackage{amsmath}
\usepackage{amsfonts}
\usepackage{epstopdf}
\usepackage[table]{xcolor}
\usepackage{matlab}

\sloppy
\epstopdfsetup{outdir=./}
\graphicspath{ {./README_images/} }

\matlabhastoc

\begin{document}

\label{T_D461CD37}
\matlabtitle{MATLAB/Simulink ユーティリティスクリプト}

\matlabtableofcontents{目次}

\vspace{1em}

\label{H_9E2D516A}
\matlabheading{目的}

\begin{itemize}
\setlength{\itemsep}{-1ex}
   \item{\begin{flushleft} MATLAB/Simulinkを使ったスクリプト処理や関数、モデル作成の効率化 \end{flushleft}}
\end{itemize}

\label{H_EA97AD53}
\matlabheading{必要なツールボックス}

\begin{itemize}
\setlength{\itemsep}{-1ex}
   \item{\begin{flushleft} MATLAB \end{flushleft}}
\end{itemize}

\label{H_694C2665}
\matlabheading{MATLABスクリプト一覧}

\begin{enumerate}
\setlength{\itemsep}{-1ex}
   \item{\begin{flushleft} \hyperref[H_828023A7]{テンプレートを使った関数や単純なスクリプトの作成(newf)} \end{flushleft}}
   \item{\begin{flushleft} \hyperref[H_2A1F15B5]{コマンドウィンドウにプログレスバーを表示(textprogressbar)} \end{flushleft}}
   \item{\begin{flushleft} \hyperref[H_4F6AB9A5]{MATLAB環境の設定(matlab\_environment\_setting)} \end{flushleft}}
\end{enumerate}

\label{H_C01E4C28}
\matlabheading{Simulinkスクリプト一覧}

\begin{enumerate}
\setlength{\itemsep}{-1ex}
   \item{\begin{flushleft} \hyperref[H_880AABF8]{モデルタブをすべて消す(allTabAndWindowClose)} \end{flushleft}}
   \item{\begin{flushleft} \hyperref[H_2C53A131]{モデルの表示大きさをすべての階層でウィンドウにフィットさせる(windowFitting)} \end{flushleft}}
\end{enumerate}


\label{H_828023A7}
\matlabheading{テンプレートを使った関数や単純なスクリプトの作成(newf)}


\vspace{1em}

\label{H_2A1F15B5}
\matlabheading{コマンドウィンドウにプログレスバーを表示(textprogressbar)}


\vspace{1em}

\label{H_4F6AB9A5}
\matlabheading{MATLAB環境の設定(matlab\_environment\_setting)}


\vspace{1em}

\label{H_880AABF8}
\matlabheading{モデルタブをすべて消す(allTabAndWindowClose)}


\vspace{1em}

\label{H_2C53A131}
\matlabheading{モデルの表示大きさをすべての階層でウィンドウにフィットさせる(windowFitting)}

\end{document}
